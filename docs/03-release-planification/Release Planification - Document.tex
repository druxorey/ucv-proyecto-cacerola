% Define the document type and font size
\documentclass[12pt]{article}

\usepackage[spanish]{babel} % Package for the Spanish language
\usepackage[utf8]{inputenc} % Package for UTF-8 character encoding
\usepackage{geometry} % Package to configure document margins
\usepackage{listings} % Package to include source code in the document
\usepackage{xcolor} % Package to define colors
\usepackage{fancyhdr} % Package to customize headers and footers
\usepackage{amsmath} % Package for advanced mathematical symbols and environments
\usepackage{amssymb} % Package for additional mathematical symbols
\usepackage{graphicx} % Package to include graphics
\usepackage{multirow} % Package to create tables with multiple rows
\usepackage{ragged2e} % Package to justify text
\usepackage{pdflscape} % Package to create landscape pages

\setlength{\parskip}{1em} % Space between paragraphs
\setlength{\parindent}{0pt} % Without indentation for paragraphs
\usepackage{enumitem} % To customize lists

% Definition of colors for source code
\definecolor{COLOR_COMMENTS}{HTML}{888888}
\definecolor{COLOR_KEYWORDS}{HTML}{8839ef}
\definecolor{COLOR_IDENTIFIERS}{HTML}{ea76cb}
\definecolor{COLOR_STRINGS}{HTML}{40a02b}

% Configuration of the listings environment to display source code
\lstset{
	frame=shadowbox,
	language=java,
	aboveskip=3mm,
	belowskip=3mm,
	xleftmargin=10mm,
	xrightmargin=10mm,
	showstringspaces=false,
	columns=flexible,
	basicstyle={\small\ttfamily},
	numbers=left,
	numberstyle=\tiny\color{COLOR_COMMENTS},
	commentstyle=\color{COLOR_COMMENTS},
	stringstyle=\color{COLOR_STRINGS},
    keywordstyle=\color{COLOR_KEYWORDS},
	morekeywords={i64},
	breaklines=true,
	breakatwhitespace=true,
	tabsize=4
}

% Configuration of document margins
\geometry{
	a4paper,
	left=25mm,
	right=25mm,
	top=25mm,
	bottom=25mm
}

\title{Proyecto Cacerola - Planificación de Lanzamiento} % Document title
\author{Druxorey} % Document author
\date{\today} % Document date

\begin{document}

\begin{titlepage}
	\centering
	\vspace{1cm}
	{\large {Universidad Central de Venezuela}\par}
	{\large {Facultad de Ciencias}\par}
	{\large {Escuela de Computación}\par}
	{\large {Ingeniería de Software}\par}
	\vspace{6cm}
	{\LARGE \textbf{Planificación de Lanzamiento \\ (Product Backlog)}\par}
	\vspace{0.25cm}
	{\Large \textbf{Entrega 3}\par}
	\vfill
	\begin{flushleft}
		{\large Profesor: Marcel Castro\par\vspace{-0.5em}}
		{\large Sección: C2\par\vspace{-0.5em}}
		{\large Equipo \#2\par\vspace{-1em}}
		\begin{itemize}
			\item Guillermo Galavís\vspace{-0.5em}
			\item José Apure\vspace{-0.5em}
			\item Renzo Morales\vspace{-0.5em}
			\item Luis David Lima\vspace{-0.5em}
		\end{itemize}
	\end{flushleft}
	\vspace{0.5cm}
	\centering
	{\large \today\par}
\end{titlepage}

\setcounter{tocdepth}{2} % Depth of the table of contents
\tableofcontents % Table of contents

\pagebreak

\section{Historias de Usuario}

\subsection{[HU001] - Solicitud de Cita para Registro}

Como usuario no registrado, quiero completar un formulario con mis datos personales para enviar una solicitud de cita a la Secretaría, con el fin de iniciar el proceso de registro físico por identificación biométrica facial para el servicio de comedor universitario.

\begin{itemize}
	\item \textbf{Prioridad:} Alta
	\item \textbf{Estimación} 5
\end{itemize}

\textbf{Criterios de Aceptación:}
\begin{itemize}
	\item \textbf{Escenario: Solicitud de cita enviada exitosamente.}
	\begin{itemize}
		\item \textbf{Dado} que el usuario no registrado ha accedido a la pantalla de solicitud de registro,
		\item \textbf{Cuando} el usuario completa todos los campos requeridos del formulario (ej. nombre, cédula de identidad, correo electrónico, número de teléfono) y hace clic en «Enviar solicitud»,
		\item \textbf{Entonces} el sistema debe validar los datos, guardar la solicitud de cita y mostrar un mensaje de confirmación al usuario indicando que su solicitud ha sido enviada y está pendiente de revisión.
	\end{itemize}

	\item \textbf{Escenario: Solicitud de cita fallida por campos incompletos o inválidos.}
	\begin{itemize}
		\item \textbf{Dado} que el usuario no registrado ha accedido a la pantalla de solicitud de registro,
		\item \textbf{Cuando} el usuario intenta enviar el formulario sin completar todos los campos obligatorios o con datos en un formato incorrecto (ej. correo electrónico inválido, cédula no numérica),
		\item \textbf{Entonces} el sistema debe mostrar mensajes de error claros para cada campo inválido o incompleto y no permitir el envío de la solicitud hasta que los errores sean corregidos.
	\end{itemize}

	\item \textbf{Escenario: Solicitud de cita con cédula ya registrada o con solicitud pendiente.}
	\begin{itemize}
		\item \textbf{Dado} que el usuario no registrado ha accedido a la pantalla de solicitud de registro,
		\item \textbf{Cuando} el usuario intenta enviar el formulario con una cédula de identidad que ya existe en el sistema (ya sea como usuario registrado o con una solicitud de cita pendiente),
		\item \textbf{Entonces} el sistema debe notificar al usuario que la cédula ya está asociada a una cuenta o solicitud existente y sugerirle que contacte a soporte técnico o intente iniciar sesión.
	\end{itemize}
\end{itemize}

\pagebreak

\subsection{[HU002] - Registro Físico de Usuario por Administrador}

Como administrador de la Secretaría, quiero validar los datos de una solicitud de registro, realizar el escaneo facial biométrico y guardar la información del usuario en la base de datos, para finalizar el proceso de registro y otorgar acceso a la aplicación.

\begin{itemize}
	\item \textbf{Prioridad:} Alta
	\item \textbf{Estimación} 13
\end{itemize}

\textbf{Criterios de Aceptación:}
\begin{itemize}
	\item \textbf{Escenario: Registro físico de usuario exitoso.}
	\begin{itemize}
		\item \textbf{Dado} que un administrador ha accedido a la gestión de solicitudes de registro,
		\item \textbf{Cuando} el administrador selecciona una solicitud pendiente, valida los datos del usuario en persona, realiza el escaneo facial biométrico y confirma el registro,
		\item \textbf{Entonces} el sistema debe guardar la información personal y los datos faciales del usuario, crear una cuenta de usuario activa en el sistema y notificar al usuario sobre su registro exitoso.
	\end{itemize}

	\item \textbf{Escenario: Rechazo de solicitud de registro por administrador.}
	\begin{itemize}
		\item \textbf{Dado} que un administrador ha accedido al módulo de gestión de solicitudes de registro,
		\item \textbf{Cuando} el administrador selecciona una solicitud pendiente y determina que los datos son inválidos o que el registro no procede por algún motivo,
		\item \textbf{Entonces} el sistema debe permitir al administrador rechazar la solicitud, opcionalmente proporcionando un motivo, y eliminar la solicitud de la lista de pendientes.
	\end{itemize}

	\item \textbf{Escenario: Error durante el escaneo facial biométrico.}
	\begin{itemize}
		\item \textbf{Dado} que un administrador está procesando una solicitud de registro y ha iniciado el escaneo facial,
		\item \textbf{Cuando} ocurre un error durante el proceso de escaneo facial (ej. falla del dispositivo, imagen no clara, no se detecta un rostro),
		\item \textbf{Entonces} el sistema debe mostrar un mensaje de error claro al administrador e indicar los pasos para reintentar el escaneo o solucionar el problema.
	\end{itemize}

	\item \textbf{Escenario: Usuario ya existente durante el proceso de registro físico.}
	\begin{itemize}
		\item \textbf{Dado} que un administrador está procesando una solicitud de registro,
		\item \textbf{Cuando} el sistema detecta que la cédula de identidad o los datos biométricos del usuario ya están asociados a una cuenta existente en la base de datos,
		\item \textbf{Entonces} el sistema debe notificar al administrador sobre la duplicidad y evitar la creación de una nueva cuenta, sugiriendo una acción como actualizar la cuenta existente si aplica.
	\end{itemize}
\end{itemize}

\pagebreak

\subsection{[HU003] - Inicio de Sesión con Credenciales}

Como usuario registrado (estudiante, profesor o empleado), quiero iniciar sesión en el sistema SGCU ingresando mi cédula de identidad y contraseña válidas, para poder acceder a las funcionalidades disponibles según mi rol.

\begin{itemize}
	\item \textbf{Prioridad:} Alta
	\item \textbf{Estimación} 3
\end{itemize}

\textbf{Criterios de Aceptación:}
\begin{itemize}
	\item \textbf{Escenario: Inicio de sesión exitoso.}
	\begin{itemize}
		\item \textbf{Dado} que el usuario registrado ha accedido a la pantalla de login,
		\item \textbf{Cuando} el usuario ingresa su cédula de identidad y contraseña válidas, y hace clic en «Iniciar Sesión»,
		\item \textbf{Entonces} el sistema debe autenticar al usuario y redirigirlo a la página principal del SGCU correspondiente a su rol (ej. vista de turnos disponibles para un usuario regular, panel de administración para un administrador).
	\end{itemize}

	\item \textbf{Escenario: Inicio de sesión fallido por credenciales inválidas.}
	\begin{itemize}
		\item \textbf{Dado} que el usuario ha accedido a la pantalla de login,
		\item \textbf{Cuando} el usuario ingresa una cédula de identidad o contraseña incorrecta, y hace clic en «Iniciar Sesión»,
		\item \textbf{Entonces} el sistema debe mostrar un mensaje de error genérico indicando que las credenciales son inválidas y mantener al usuario en la pantalla de login.
	\end{itemize}

	\item \textbf{Escenario: Intentos de inicio de sesión fallidos repetidos.}
	\begin{itemize}
		\item \textbf{Dado} que el usuario ha intentado iniciar sesión varias veces con credenciales incorrectas,
		\item \textbf{Cuando} el número de intentos fallidos excede un límite predefinido,
		\item \textbf{Entonces} el sistema debe bloquear temporalmente la cuenta del usuario o requerir un método de verificación adicional, y notificar al usuario sobre el bloqueo.
	\end{itemize}
\end{itemize}

\pagebreak

\subsection{[HU004] - Recuperación de Contraseña}

Como usuario registrado, quiero poder recuperar o restablecer mi contraseña en caso de olvido, para poder acceder nuevamente a mi cuenta.

\begin{itemize}
	\item \textbf{Prioridad:} Alta
	\item \textbf{Estimación} 8
\end{itemize}

\textbf{Criterios de Aceptación:}
\begin{itemize}
	\item \textbf{Escenario: Solicitud de recuperación de contraseña exitosa.}
	\begin{itemize}
		\item \textbf{Dado} que el usuario ha olvidado su contraseña y ha accedido a la opción «Olvidé mi contraseña»,
		\item \textbf{Cuando} el usuario ingresa su cédula de identidad o correo electrónico registrado y el sistema lo valida,
		\item \textbf{Entonces} el sistema debe enviar un enlace o código de recuperación a su correo electrónico o número de teléfono registrado y mostrar un mensaje de confirmación en pantalla.
	\end{itemize}

	\item \textbf{Escenario: Restablecimiento de contraseña exitoso.}
	\begin{itemize}
		\item \textbf{Dado} que el usuario ha recibido el enlace o código de recuperación y lo ha utilizado para acceder a la página de restablecimiento,
		\item \textbf{Cuando} el usuario ingresa su nueva contraseña y la confirma, cumpliendo con los requisitos de seguridad establecidos (ej. longitud mínima, caracteres especiales, no ser igual a las últimas N contraseñas),
		\item \textbf{Entonces} el sistema debe actualizar la contraseña del usuario de forma segura y permitirle iniciar sesión con las nuevas credenciales.
	\end{itemize}

	\item \textbf{Escenario: Restablecimiento de contraseña fallido por enlace/código \\ inválido o expirado.}
	\begin{itemize}
		\item \textbf{Dado} que el usuario está intentando restablecer su contraseña,
		\item \textbf{Cuando} el usuario intenta usar un enlace o código de recuperación que ya ha sido utilizado, ha expirado o es inválido,
		\item \textbf{Entonces} el sistema debe mostrar un mensaje de error indicando que el enlace o código no es válido o ha expirado, y sugerirle que genere una nueva solicitud de recuperación si es necesario.
	\end{itemize}
\end{itemize}

\pagebreak

\subsection{[HU005] - Gestión de Usuarios Registrados}

Como administrador de la Secretaría, quiero poder consultar, editar o desactivar las cuentas de usuarios ya registrados, para mantener la base de datos de usuarios actualizada y segura.

\begin{itemize}
	\item \textbf{Prioridad:} Alta
	\item \textbf{Estimación} 8
\end{itemize}

\textbf{Criterios de Aceptación:}
\begin{itemize}
	\item \textbf{Escenario: Consulta de usuarios registrados.}
	\begin{itemize}
		\item \textbf{Dado} que el administrador de la Secretaría ha accedido al módulo de gestión de usuarios,
		\item \textbf{Cuando} el administrador solicita ver la lista de usuarios,
		\item \textbf{Entonces} el sistema debe mostrar una lista paginada y filtrable de todos los usuarios registrados con sus datos principales (ej. cédula de identidad, nombre completo, correo electrónico y estado de la cuenta).
	\end{itemize}

	\item \textbf{Escenario: Edición exitosa de datos de usuario.}
	\begin{itemize}
		\item \textbf{Dado} que el administrador ha consultado un usuario específico y ha accedido a su perfil de edición,
		\item \textbf{Cuando} el administrador modifica los datos del usuario (ej. información de contacto, rol o tipo de usuario) y guarda los cambios,
		\item \textbf{Entonces} el sistema debe actualizar la información del usuario en la base de datos y mostrar un mensaje de confirmación al administrador.
	\end{itemize}

	\item \textbf{Escenario: Desactivación exitosa de cuenta de usuario.}
	\begin{itemize}
		\item \textbf{Dado} que el administrador ha consultado un usuario específico,
		\item \textbf{Cuando} el administrador selecciona la opción para desactivar la cuenta del usuario y confirma la acción,
		\item \textbf{Entonces} el sistema debe cambiar el estado de la cuenta del usuario a «inactivo», impidiendo su inicio de sesión, y la cuenta debe seguir siendo visible para el administrador con su estado actualizado.
	\end{itemize}

	\item \textbf{Escenario: Reactivación exitosa de cuenta de usuario.}
	\begin{itemize}
		\item \textbf{Dado} que el administrador ha consultado un usuario con la cuenta desactivada,
		\item \textbf{Cuando} el administrador selecciona la opción para reactivar la cuenta del usuario y confirma la acción,
		\item \textbf{Entonces} el sistema debe cambiar el estado de la cuenta del usuario a «activo», permitiendo su inicio de sesión nuevamente.
	\end{itemize}

	\item \textbf{Escenario: Error al editar/desactivar/reactivar por datos inválidos o permisos insuficientes.}
	\begin{itemize}
		\item \textbf{Dado} que el administrador intenta editar, desactivar o reactivar una cuenta de usuario,
		\item \textbf{Cuando} los datos ingresados para la edición son inválidos, o el administrador no tiene los permisos necesarios para realizar la acción,
		\item \textbf{Entonces} el sistema debe mostrar un mensaje de error específico que indique la causa del fallo y no ejecutar la acción.
	\end{itemize}
\end{itemize}

\pagebreak

\subsection{[HU006] - Gestión de Horarios y Asignación de Menús}

Como administrador del comedor, quiero agregar, modificar y visualizar los horarios de atención y asignar los menús específicos para cada turno y día de la semana, para asegurar que los usuarios conozcan la disponibilidad del servicio y las opciones de comida.

\begin{itemize}
	\item \textbf{Prioridad:} Alta
	\item \textbf{Estimación} 8
\end{itemize}

\textbf{Criterios de Aceptación:}
\begin{itemize}
	\item \textbf{Escenario: Creación exitosa de un nuevo horario de atención.}
	\begin{itemize}
		\item \textbf{Dado} que el administrador del comedor ha accedido al módulo de gestión de horarios,
		\item \textbf{Cuando} el administrador especifica un nuevo horario (ej. día de la semana, hora de inicio, hora de fin, tipo de comida como «Desayuno», «Almuerzo» o «Cena») y lo guarda,
		\item \textbf{Entonces} el sistema debe registrar el nuevo horario y mostrarlo en la lista de horarios disponibles para asignación de menús.
	\end{itemize}

	\item \textbf{Escenario: Asignación exitosa de menú a un turno.}
	\begin{itemize}
		\item \textbf{Dado} que el administrador ha configurado los horarios y ha accedido a la gestión de menús,
		\item \textbf{Cuando} el administrador selecciona un turno específico (día y hora) y le asigna un menú predefinido o crea uno nuevo con sus respectivos platos para ese turno,
		\item \textbf{Entonces} el sistema debe asociar el menú al turno seleccionado y este debe ser visible para los usuarios en la vista de menús semanales.
	\end{itemize}

	\item \textbf{Escenario: Modificación exitosa de un horario o menú asignado.}
	\begin{itemize}
		\item \textbf{Dado} que el administrador del comedor ha accedido a un horario o menú ya existente,
		\item \textbf{Cuando} el administrador realiza cambios en el horario (ej. hora de fin) o en los platos de un menú asignado y guarda,
		\item \textbf{Entonces} el sistema debe actualizar la información y reflejar los cambios en la vista de los usuarios.
	\end{itemize}

	\item \textbf{Escenario: Error al agregar/modificar por datos inválidos o duplicados.}
	\begin{itemize}
		\item \textbf{Dado} que el administrador intenta agregar o modificar un horario/menú,
		\item \textbf{Cuando} el administrador ingresa datos inválidos (ej. hora de fin anterior a hora de inicio) o intenta crear un horario que se solapa con uno existente para el mismo tipo de comida,
		\item \textbf{Entonces} el sistema debe mostrar un mensaje de error específico y no permitir la operación hasta que se corrijan los datos.
	\end{itemize}
\end{itemize}

\pagebreak

\subsection{[HU007] - Visualización del Menú Semanal y Selección de Comidas}

Como usuario del SGCU (estudiante, profesor o personal obrero), quiero poder ver los menús disponibles para la semana actual por turnos, incluyendo los platos, y poder seleccionar las comidas deseadas, para organizar mis consumos en el comedor.

\begin{itemize}
	\item \textbf{Prioridad:} Alta
	\item \textbf{Estimación} 8
\end{itemize}

\textbf{Criterios de Aceptación:}
\begin{itemize}
	\item \textbf{Escenario: Visualización exitosa del menú semanal.}
	\begin{itemize}
		\item \textbf{Dado} que el usuario ha iniciado sesión en el SGCU y ha accedido a la sección de menús o a la vista principal,
		\item \textbf{Cuando} el sistema carga la vista del menú semanal,
		\item \textbf{Entonces} el sistema debe mostrar claramente los días de la semana, los turnos disponibles (ej. «Desayuno», «Almuerzo», «Cena») y los menús asignados para cada uno, incluyendo los platos y una descripción básica de los ingredientes.
	\end{itemize}

	\item \textbf{Escenario: Selección exitosa de una comida.}
	\begin{itemize}
		\item \textbf{Dado} que el usuario está visualizando el menú semanal,
		\item \textbf{Cuando} el usuario selecciona una o varias comidas de un turno específico,
		\item \textbf{Entonces} el sistema debe marcar visiblemente las comidas seleccionadas y añadir estos ítems a una lista temporal de selección para el pago, reflejando la elección del usuario.
	\end{itemize}

	\item \textbf{Escenario: Deselección de una comida.}
	\begin{itemize}
		\item \textbf{Dado} que el usuario ha seleccionado previamente una comida,
		\item \textbf{Cuando} el usuario deselecciona una comida que ya había elegido,
		\item \textbf{Entonces} el sistema debe eliminar la marca visual de selección y retirar el ítem de la lista temporal para el pago.
	\end{itemize}

	\item \textbf{Escenario: Visualización de detalles de plato.}
	\begin{itemize}
		\item \textbf{Dado} que el usuario está viendo un plato en el menú,
		\item \textbf{Cuando} el usuario interactúa con un plato (ej. haciendo clic o tocando),
		\item \textbf{Entonces} el sistema debe mostrar detalles adicionales del plato, como por ejemplo los ingredientes para prepararlo.
	\end{itemize}
\end{itemize}

\pagebreak

\subsection{[HU008] - Gestión de Ítems Seleccionados para Pago}

Como usuario del SGCU, quiero que el sistema me notifique visiblemente los ítems que he seleccionado para el pago en un turno, para tener un resumen claro antes de proceder a la transacción.

\begin{itemize}
	\item \textbf{Prioridad:} Alta
	\item \textbf{Estimación} 3
\end{itemize}

\textbf{Criterios de Aceptación:}
\begin{itemize}
	\item \textbf{Escenario: Notificación visual de ítems seleccionados al añadir.}
	\begin{itemize}
		\item \textbf{Dado} que el usuario ha seleccionado una o más comidas en la sección de menú semanal,
		\item \textbf{Cuando} el sistema registra la selección de un ítem para el pago,
		\item \textbf{Entonces} el sistema debe actualizar de forma visible un indicador (ej. icono de carrito con un número, notificación en la parte superior derecha de la pantalla) que muestre la cantidad de ítems seleccionados para el pago.
	\end{itemize}

	\item \textbf{Escenario: Notificación visual de ítems seleccionados al eliminar.}
	\begin{itemize}
		\item \textbf{Dado} que el usuario ha seleccionado previamente ítems para el pago,
		\item \textbf{Cuando} el usuario deselecciona una comida de su lista,
		\item \textbf{Entonces} el sistema debe actualizar de forma visible el indicador de ítems seleccionados, reduciendo la cantidad mostrada.
	\end{itemize}

	\item \textbf{Escenario: Resumen claro de ítems seleccionados en la sección de pago.}
	\begin{itemize}
		\item \textbf{Dado} que el usuario ha seleccionado ítems para el pago y ha accedido a la sección de pago,
		\item \textbf{Cuando} el sistema presenta el resumen de la compra,
		\item \textbf{Entonces} el sistema debe mostrar una lista clara y detallada de cada ítem seleccionado, incluyendo su nombre, el precio unitario, y el turno al que corresponde, así como un subtotal antes de la confirmación final de la transacción.
	\end{itemize}

	\item \textbf{Escenario: Consistencia de la selección a través de la navegación.}
	\begin{itemize}
		\item \textbf{Dado} que el usuario ha seleccionado ítems para el pago,
		\item \textbf{Cuando} el usuario navega a otras secciones de la aplicación y luego regresa a la vista del menú o a la sección de pago,
		\item \textbf{Entonces} la selección de ítems debe persistir y el indicador visual debe reflejar correctamente la cantidad y los detalles de los ítems aún seleccionados.
	\end{itemize}
\end{itemize}

\pagebreak

\subsection{[HU009] - Pago Seguro de Comida y Generación de Factura QR}

Como usuario del SGCU, quiero seleccionar mi menú, confirmar mi identidad mediante reconocimiento facial, realizar un pago seguro y obtener una factura con un código QR único, para poder consumir mi comida de manera eficiente y segura.

\begin{itemize}
	\item \textbf{Prioridad:} Alta
	\item \textbf{Estimación} 13
\end{itemize}

\textbf{Criterios de Aceptación:}
\begin{itemize}
	\item \textbf{Escenario: Proceso de pago y generación de factura QR exitosos.}
	\begin{itemize}
		\item \textbf{Dado} que el usuario ha seleccionado ítems para el pago y ha accedido a la sección de pago,
		\item \textbf{Cuando} el usuario inicia el proceso de pago, el sistema solicita y valida su identidad mediante reconocimiento facial, presenta un resumen claro de la compra, y el usuario procede con la doble confirmación,
		\item \textbf{Entonces} el sistema debe deducir el monto del monedero virtual del usuario, generar una factura digital con un código QR único y los datos pertinentes al pago, mostrar una notificación de pago exitoso (ej. «¡Pago realizado con éxito! Puedes ver tu factura en la sección de facturas») y redirigir al usuario a la pantalla principal.
	\end{itemize}

	\item \textbf{Escenario: Pago fallido por saldo insuficiente.}
	\begin{itemize}
		\item \textbf{Dado} que el usuario ha seleccionado ítems para el pago y ha accedido a la sección de pago,
		\item \textbf{Cuando} el usuario intenta realizar el pago, pero su saldo en el monedero virtual es insuficiente para cubrir el total de la compra,
		\item \textbf{Entonces} el sistema debe mostrar un mensaje de error claro (ej. «Saldo insuficiente. Por favor, recargue su monedero virtual») y no proceder con el pago, sugiriendo opciones para recargar el monedero.
	\end{itemize}

	\item \textbf{Escenario: Pago fallido por reconocimiento facial no validado.}
	\begin{itemize}
		\item \textbf{Dado} que el usuario ha seleccionado ítems para el pago y ha accedido a la sección de pago,
		\item \textbf{Cuando} el sistema solicita el reconocimiento facial y no puede validar la identidad del usuario (ej. rostro no reconocido, falla técnica del sensor, no coincide con los datos registrados),
		\item \textbf{Entonces} el sistema debe mostrar un mensaje de error de validación facial (ej. «Reconocimiento facial fallido. Intente nuevamente o contacte a la Secretaría») y no proceder con el pago, permitiendo reintentar el reconocimiento o cancelar la operación.
	\end{itemize}

	\item \textbf{Escenario: Cancelación de pago antes de la confirmación final.}
	\begin{itemize}
		\item \textbf{Dado} que el usuario ha iniciado el proceso de pago y el resumen está siendo presentado para doble confirmación,
		\item \textbf{Cuando} el usuario decide cancelar la operación antes de la confirmación final del procesamiento,
		\item \textbf{Entonces} el sistema debe detener el proceso de pago y devolver al usuario a la pantalla anterior o a la sección de selección de ítems, sin realizar cargos.
	\end{itemize}

	\item \textbf{Escenario: Fallo técnico durante el procesamiento del pago.}
	\begin{itemize}
		\item \textbf{Dado} que el usuario ha confirmado el pago y el reconocimiento facial ha sido exitoso,
		\item \textbf{Cuando} ocurre un fallo técnico inesperado durante el procesamiento del pago (ej. error de conexión con la base de datos, falla del servidor del monedero virtual),
		\item \textbf{Entonces} el sistema debe notificar al usuario sobre el error (ej. «Ha ocurrido un error inesperado al procesar el pago. Por favor, intente de nuevo o contacte a soporte»), asegurar que no se realice ningún cargo o que cualquier cargo parcial sea reembolsado, y registrar el incidente para revisión del administrador.
	\end{itemize}
\end{itemize}

\textbf{Requisitos No Funcionales:}
\begin{itemize}
	\item \textbf{Disponibilidad de Cancelación:} El sistema debe permitir al usuario cancelar el pago durante un período de 3 segundos después de la confirmación final, antes de que el proceso de deducción del saldo sea irreversible.
	\item \textbf{Presentación de la Información de Pago:} El sistema debe garantizar que, previo a la confirmación de la transacción, se muestre al usuario de forma clara y completa un resumen de los ítems seleccionados, el total a pagar, el saldo actual de su monedero virtual y el saldo restante estimado después del pago.
\end{itemize}

\pagebreak

\subsection{[HU010] - Visualización y Canje de Facturas QR}

Como usuario del SGCU, quiero poder ver una lista de mis facturas con QR de pagos realizados, y que el personal del comedor pueda escanearlas para validar mi pedido y entregar la comida, para un proceso de retiro ágil.

\begin{itemize}
	\item \textbf{Prioridad:} Alta
	\item \textbf{Estimación} 8
\end{itemize}

\textbf{Criterios de Aceptación:}
\begin{itemize}
	\item \textbf{Escenario: Visualización exitosa de lista de facturas.}
	\begin{itemize}
		\item \textbf{Dado} que el usuario ha iniciado sesión y navega a la sección de «Mis Facturas» dentro de la aplicación,
		\item \textbf{Cuando} el sistema carga la vista de facturas,
		\item \textbf{Entonces} el sistema debe mostrar una lista de todas las últimas facturas de pagos realizados por el usuario, con cada entrada mostrando al menos la fecha de compra, el total pagado, y un indicador visual de su estado.
	\end{itemize}

	\item \textbf{Escenario: Visualización de factura QR detallada.}
	\begin{itemize}
		\item \textbf{Dado} que el usuario está viendo la lista de facturas,
		\item \textbf{Cuando} el usuario selecciona una factura específica para ver sus detalles,
		\item \textbf{Entonces} el sistema debe mostrar la factura completa, incluyendo los ítems comprados, el total pagado, la fecha y hora de la compra, y un código QR único de alta resolución junto con otros datos pertinentes al pago, listo para ser escaneado.
	\end{itemize}

	\item \textbf{Escenario: Canje exitoso de factura por personal del comedor.}
	\begin{itemize}
		\item \textbf{Dado} que un usuario presenta una factura con QR al personal del comedor y este tiene acceso al sistema de escaneo de QR,
		\item \textbf{Cuando} el personal escanea el código QR de la factura, el sistema lo valida (comprueba que sea una factura válida, no canjeada y perteneciente al turno actual), y el personal confirma la entrega de la comida,
		\item \textbf{Entonces} el sistema debe marcar la factura como «Canjeada», registrar la transacción de canje, y luego la factura debe ser eliminada de la lista de facturas visibles para el usuario.
	\end{itemize}

	\item \textbf{Escenario: Intento de canje de factura ya canjeada o inválida.}
	\begin{itemize}
		\item \textbf{Dado} que un usuario presenta una factura con QR al personal del comedor,
		\item \textbf{Cuando} el personal escanea un código QR que corresponde a una factura ya canjeada, una factura inválida, una factura expirada o un QR corrupto,
		\item \textbf{Entonces} el sistema debe mostrar un mensaje de error claro al personal (ej. «Factura ya canjeada», «Código QR inválido» o «Factura expirada») y no permitir el canje.
	\end{itemize}

	\item \textbf{Escenario: Factura con QR no verificada en el plazo de una semana.}
	\begin{itemize}
		\item \textbf{Dado} que ha transcurrido una semana desde la emisión de una factura con QR,
		\item \textbf{Cuando} el sistema realiza una verificación automática de las facturas no canjeadas,
		\item \textbf{Entonces} el sistema debe reembolsar automáticamente el dinero correspondiente a esa factura al monedero virtual del usuario y emitir una advertencia o notificación al usuario (ej. «Advertencia: Su factura [ID de Factura] del [Fecha de Compra] no fue canjeada. El dinero correspondiente ha sido reembolsado a su monedero virtual»).
	\end{itemize}
\end{itemize}

\pagebreak

\subsection{[HU011] - Reembolso Automático de Facturas No Canjeadas}

Como usuario del SGCU, quiero que el sistema reembolse automáticamente el dinero a mi monedero virtual y me notifique si una factura con QR no ha sido canjeada en un plazo de una semana, para evitar pérdidas de saldo por servicios no utilizados.

\begin{itemize}
	\item \textbf{Prioridad:} Alta
	\item \textbf{Estimación} 8
\end{itemize}

\textbf{Criterios de Aceptación:}
\begin{itemize}
	\item \textbf{Escenario: Reembolso automático exitoso de factura no canjeada.}
	\begin{itemize}
		\item \textbf{Dado} que existe una factura con QR emitida hace más de una semana que no ha sido canjeada por el personal del comedor,
		\item \textbf{Cuando} el proceso automático de verificación de facturas se ejecuta (ej. diariamente a una hora predefinida),
		\item \textbf{Entonces} el sistema debe identificar la factura, reembolsar el monto total correspondiente al monedero virtual del usuario, y cambiar el estado de la factura a «Reembolsada» en el registro.
	\end{itemize}

	\item \textbf{Escenario: Notificación de reembolso enviada al usuario.}
	\begin{itemize}
		\item \textbf{Dado} que una factura ha sido reembolsada automáticamente a un usuario,
		\item \textbf{Cuando} el sistema procesa el reembolso,
		\item \textbf{Entonces} el sistema debe enviar una notificación al usuario (ej. vía correo electrónico o notificación push en la aplicación) informándole sobre el reembolso, incluyendo el ID de la factura, el monto reembolsado y la razón (factura no canjeada en el plazo establecido).
	\end{itemize}

	\item \textbf{Escenario: Facturas canjeadas o dentro del plazo no son reembolsadas.}
	\begin{itemize}
		\item \textbf{Dado} que existen facturas que ya han sido canjeadas o que fueron emitidas hace menos de una semana,
		\item \textbf{Cuando} el proceso automático de verificación de facturas se ejecuta,
		\item \textbf{Entonces} el sistema no debe realizar ningún reembolso para esas facturas, ya que cumplen con los criterios de no-reembolso.
	\end{itemize}

	\item \textbf{Escenario: Fallo en el proceso de reembolso.}
	\begin{itemize}
		\item \textbf{Dado} que el sistema intenta reembolsar una factura,
		\item \textbf{Cuando} ocurre un fallo técnico durante el proceso de reembolso (ej. error de conexión con el servicio del monedero virtual, inconsistencia de datos que impida el reembolso),
		\item \textbf{Entonces} el sistema debe registrar el incidente con detalles del error para que sea revisado por un administrador, y si es posible, reintentar el reembolso en un momento posterior, asegurando la integridad del saldo del usuario.
	\end{itemize}
\end{itemize}

\pagebreak

\subsection{[HU012] - Consulta de Saldo y Movimientos del Monedero Virtual}

Como usuario del SGCU (estudiante, profesor o personal obrero), quiero poder consultar el saldo actual de mi monedero virtual y ver un historial de mis gastos y recargas, para gestionar mi presupuesto y saber cuánto dinero tengo disponible para el servicio de comedor.

\begin{itemize}
	\item \textbf{Prioridad:} Alta
	\item \textbf{Estimación} 5
\end{itemize}

\textbf{Criterios de Aceptación:}
\begin{itemize}
	\item \textbf{Escenario: Visualización exitosa del saldo actual.}
	\begin{itemize}
		\item \textbf{Dado} que el usuario ha iniciado sesión en el SGCU y ha accedido a la sección de su monedero virtual (ej. a través de un icono o menú),
		\item \textbf{Cuando} el sistema carga la vista del monedero,
		\item \textbf{Entonces} el sistema debe mostrar de forma clara y destacada el saldo actual del monedero virtual del usuario.
	\end{itemize}

	\item \textbf{Escenario: Visualización exitosa del historial de movimientos.}
	\begin{itemize}
		\item \textbf{Dado} que el usuario ha accedido a la sección de su monedero virtual,
		\item \textbf{Cuando} el usuario solicita ver el historial de transacciones,
		\item \textbf{Entonces} el sistema debe mostrar una lista cronológica de todos los movimientos del monedero (recargas y gastos), incluyendo la fecha y hora, una descripción de la transacción, el monto y el tipo de movimiento (ej. «Ingreso», «Egreso»).
	\end{itemize}
\end{itemize}

\pagebreak

\subsection{[HU013] - Visualización de Datos para Recarga de Monedero Virtual}

Como usuario del SGCU (estudiante, profesor o personal obrero), quiero visualizar los datos de cuenta necesarios para afiliar mi monedero virtual a una aplicación bancaria y realizar pagos móviles, para poder recargar saldo de forma conveniente y anticipada para el servicio de comedor.

\begin{itemize}
	\item \textbf{Prioridad:} Alta
	\item \textbf{Estimación} 3
\end{itemize}

\textbf{Criterios de Aceptación:}
\begin{itemize}
	\item \textbf{Escenario: Visualización exitosa de datos de recarga.}
	\begin{itemize}
		\item \textbf{Dado} que el usuario ha iniciado sesión y navega a la sección de recarga de su monedero virtual,
		\item \textbf{Cuando} el sistema carga la pantalla de recarga por pago móvil,
		\item \textbf{Entonces} el sistema debe mostrar claramente todos los datos de cuenta necesarios para realizar un pago móvil (ej. número de teléfono del beneficiario, banco destino, cédula o RIF del titular, nombre del titular de la cuenta del SGCU).
	\end{itemize}
\end{itemize}

\pagebreak

\subsection{[HU014] - Gestión de Costos Operacionales y CCB}

Como administrador, quiero registrar y consultar los costos operativos (fijos y variables), el número de bandejas y el porcentaje de merma, para calcular el Costo Cubierto de la Bandeja (CCB) de manera precisa.

\begin{itemize}
	\item \textbf{Prioridad:} Alta
	\item \textbf{Estimación} 8
\end{itemize}

\textbf{Criterios de Aceptación:}
\begin{itemize}
	\item \textbf{Escenario: Registro exitoso de costos operativos y datos de cálculo.}
	\begin{itemize}
		\item \textbf{Dado} que el administrador ha accedido al módulo de gestión de costos operacionales,
		\item \textbf{Cuando} el administrador ingresa los costos operativos (ej. alquiler, salarios, insumos, servicios), el número total de bandejas servidas y el porcentaje de merma para un período específico (ej. diario o semanal), y guarda la información,
		\item \textbf{Entonces} el sistema debe registrar estos datos de forma persistente en la base de datos y utilizarlos como base para el cálculo del CCB.
	\end{itemize}

	\item \textbf{Escenario: Cálculo y visualización precisa del CCB.}
	\begin{itemize}
		\item \textbf{Dado} que el sistema tiene datos de costos operativos, número de bandejas y porcentaje de merma registrados para un período,
		\item \textbf{Cuando} el administrador solicita ver el CCB para ese período o cuando los datos son actualizados,
		\item \textbf{Entonces} el sistema debe calcular el CCB utilizando una fórmula predefinida y mostrar el resultado de manera clara, con la unidad monetaria correspondiente.
	\end{itemize}

	\item \textbf{Escenario: Validación de datos de entrada.}
	\begin{itemize}
		\item \textbf{Dado} que el administrador intenta registrar o actualizar datos de costos, número de bandejas o porcentaje de merma,
		\item \textbf{Cuando} el administrador ingresa valores inválidos (ej. porcentajes negativos o mayores a 100, texto en campos numéricos, números de bandejas en cero para un período con costos),
		\item \textbf{Entonces} el sistema debe mostrar un mensaje de error apropiado y no permitir el registro o actualización hasta que los datos sean corregidos.
	\end{itemize}
\end{itemize}

\pagebreak

\subsection{[HU015] - Configuración de Tarifas Diferenciadas}

Como administrador, quiero configurar los porcentajes de las tarifas diferenciadas para estudiantes, profesores y empleados, para asegurar la aplicación correcta de los costos del servicio según el tipo de usuario.

\begin{itemize}
	\item \textbf{Prioridad:} Alta
	\item \textbf{Estimación} 5
\end{itemize}

\textbf{Criterios de Aceptación:}
\begin{itemize}
	\item \textbf{Escenario: Configuración exitosa de porcentajes de tarifas.}
	\begin{itemize}
		\item \textbf{Dado} que el administrador ha accedido al módulo de configuración de tarifas diferenciadas,
		\item \textbf{Cuando} el administrador ingresa porcentajes válidos para las tarifas de estudiantes, profesores y empleados, y guarda la configuración,
		\item \textbf{Entonces} el sistema debe registrar los nuevos porcentajes de forma persistente y utilizarlos para los cálculos futuros de las tarifas aplicables a cada tipo de usuario.
	\end{itemize}

	\item \textbf{Escenario: Validación de rangos al configurar tarifas.}
	\begin{itemize}
		\item \textbf{Dado} que el administrador está configurando las tarifas diferenciadas,
		\item \textbf{Cuando} el administrador intenta guardar porcentajes que están fuera de los rangos predefinidos,
		\item \textbf{Entonces} el sistema debe mostrar un mensaje de error específico para cada porcentaje inválido y no permitir el guardado hasta que los valores estén dentro de los rangos permitidos.
	\end{itemize}

	\item \textbf{Escenario: Visualización de tarifas configuradas.}
	\begin{itemize}
		\item \textbf{Dado} que el administrador ha accedido al módulo de configuración de tarifas,
		\item \textbf{Cuando} el sistema carga la pantalla de configuración de tarifas,
		\item \textbf{Entonces} el sistema debe mostrar los porcentajes de tarifas actualmente configurados para cada tipo de usuario (estudiante, profesor, empleado).
	\end{itemize}

	\item \textbf{Escenario: Aplicación de tarifas a pagos.}
	\begin{itemize}
		\item \textbf{Dado} que las tarifas diferenciadas han sido configuradas por el administrador,
		\item \textbf{Cuando} un usuario de tipo estudiante, profesor o empleado procede a pagar un menú,
		\item \textbf{Entonces} el sistema debe calcular automáticamente el costo del menú para ese usuario aplicando el porcentaje correspondiente configurado sobre el Costo Cubierto de la Bandeja (CCB) actual y deducir ese monto de su monedero virtual.
	\end{itemize}
\end{itemize}

\pagebreak

\subsection{[HU016] - Generación y Exportación de Reportes Operacionales}

Como administrador, quiero generar reportes de datos del comedor para un período seleccionado (diario, semanal o mensual) y exportarlos en formato de hoja de cálculo, para analizar la demanda, el consumo, la distribución de ingresos y planificar mejor los recursos.

\begin{itemize}
	\item \textbf{Prioridad:} Alta
	\item \textbf{Estimación} 8
\end{itemize}

\textbf{Criterios de Aceptación:}
\begin{itemize}
	\item \textbf{Escenario: Generación exitosa de reportes por período.}
	\begin{itemize}
		\item \textbf{Dado} que el administrador ha accedido al módulo de reportes operacionales,
		\item \textbf{Cuando} el administrador selecciona un tipo de reporte (ej. demanda de menús, consumo por usuario, distribución de ingresos) y un período (diario, semanal o mensual),
		\item \textbf{Entonces} el sistema debe generar el reporte mostrando los datos relevantes y agregados para el período seleccionado, presentándolos en una interfaz legible.
	\end{itemize}

	\item \textbf{Escenario: Exportación exitosa de reportes a hoja de cálculo.}
	\begin{itemize}
		\item \textbf{Dado} que se ha generado un reporte en la interfaz,
		\item \textbf{Cuando} el administrador selecciona la opción de exportar el reporte,
		\item \textbf{Entonces} el sistema debe exportar los datos del reporte a un formato de hoja de cálculo común (ej. CSV o XLSX) y permitir al administrador descargar el archivo en su dispositivo.
	\end{itemize}

	\item \textbf{Escenario: Contenido preciso y relevante de los reportes.}
	\begin{itemize}
		\item \textbf{Dado} que se ha generado y/o exportado un reporte (demanda, consumo, distribución de ingresos),
		\item \textbf{Cuando} el administrador revisa el contenido del reporte,
		\item \textbf{Entonces} el reporte debe contener datos precisos, consistentes y relevantes para el tipo de reporte y período seleccionado (ej. el reporte de demanda debe mostrar el número de comidas solicitadas por plato y turno, el de consumo por usuario debe detallar las comidas consumidas por cada individuo, y el de ingresos debe desglosar las tarifas aplicadas y su distribución).
	\end{itemize}

	\item \textbf{Escenario: Notificación de ausencia de datos para el período.}
	\begin{itemize}
		\item \textbf{Dado} que el administrador selecciona un período para el cual no hay datos disponibles para generar un reporte,
		\item \textbf{Cuando} el sistema intenta generar el reporte,
		\item \textbf{Entonces} el sistema debe mostrar un mensaje claro informando que no hay datos disponibles para el período seleccionado y no generar un reporte vacío o con errores.
	\end{itemize}
\end{itemize}

\pagebreak

\subsection{Requisitos No Funcionales Detallados para Precios y Costos}

\textbf{Cálculo del Costo Cubierto de la Bandeja (CCB):}

\begin{itemize}
	\item \textbf{Fórmula de Cálculo:} El sistema debe implementar la fórmula exacta para el CCB: $CCB=[(CF+CV)/NB]^{*}(1+\%Merma)$.
	\item \textbf{Componentes del Cálculo:}
	\begin{itemize}
		\item \textbf{Costos Fijos (CF):} El sistema debe permitir la entrada y gestión de los costos fijos totales del servicio, que incluyen la mano de obra del personal de cocina y administrativo, el mantenimiento de equipos e instalaciones, entre otros.
		\item \textbf{Costos Variables (CV):} El sistema debe permitir la entrada y gestión de los costos variables totales por servicio, que fluctúan con el número de bandejas servidas, incluyendo el precio de los insumos (proteínas, lípidos y carbohidratos, considerando la distribución típica y las porciones recomendadas por el Instituto Nacional de Nutrición de Venezuela), los materiales de empaque y limpieza, y la energía consumida (electricidad, gas, etc.).
		\item \textbf{Número de Bandejas (NB):} El sistema debe considerar el número de bandejas proyectadas o servidas en un período para el cálculo del CCB.
		\item \textbf{Porcentaje de Merma (\%\_Merma):} El sistema debe incorporar un factor de desecho o merma inherente a la manipulación y preparación de los alimentos, expresado como porcentaje.
	\end{itemize}
	\item \textbf{Actualización de Componentes:} El sistema debe permitir a los administradores actualizar los valores de los costos fijos, costos variables y el porcentaje de merma para recalcular el CCB periódicamente.
\end{itemize}

\textbf{Distribución de Ingresos (Contabilización):}

\begin{itemize}
	\item \textbf{División de Pagos:} El sistema debe registrar y categorizar los ingresos obtenidos por el pago de las tarifas, distribuyendo un porcentaje (entre 25\% y 30\%) a la ganancia del concesionario y el porcentaje restante a los ingresos propios del servicio del comedor universitario.
	\item \textbf{Reporte de Distribución:} El sistema debe poder generar reportes que muestren esta distribución de ingresos para fines administrativos y de reinversión.
\end{itemize}

\pagebreak

\textbf{Aplicación de Tarifas Diferenciadas:}

\begin{itemize}
	\item \textbf{Estructura de Tarifas:} El sistema debe aplicar una estructura de tarifas diferenciadas basada en el tipo de usuario y el CCB.
	\item \textbf{Tarifa para Estudiantes:} La tarifa para estudiantes debe ser calculada y aplicada como un porcentaje entre el 20\% y el 30\% del CCB.
	\item \textbf{Tarifa para Profesores:} La tarifa para profesores debe ser calculada y aplicada como un porcentaje entre el 70\% y el 90\% del CCB.
	\item \textbf{Tarifa para Empleados:} La tarifa para empleados debe ser calculada y aplicada como un porcentaje entre el 90\% y el 110\% del CCB.
	\item \textbf{Asignación Automática:} El sistema debe asignar automáticamente la tarifa correcta al usuario al momento de la verificación de saldo y el descuento, basándose en su tipo de usuario registrado.
	\item \textbf{Flexibilidad de Configuración:} El sistema debe permitir a los administradores configurar los porcentajes exactos dentro de los rangos especificados para cada tipo de usuario (20-30\%, 70-90\%, 90-110\%).
\end{itemize}

\pagebreak

\section{Planificación del Sprint}

\subsection{Planificación del Sprint 1 (30/06/2025 - 06/07/2025)}

\subsubsection{Objetivos del Sprint 1}
\begin{itemize}
	\item Establecer la infraestructura inicial del sistema (base de datos, configuración de proyecto).
	\item Implementar la funcionalidad base de inicio de sesión (HU001).
	\item Desarrollar la funcionalidad de solicitud de cita para registro de usuarios (HU002).
	\item Definir los roles básicos de usuario y administrador.
\end{itemize}

\subsubsection{Tareas del Sprint 1}
\textbf{Tareas de Infraestructura y Configuración}
\begin{itemize}
	\item Configuración inicial del proyecto (estructura de directorios, configuración Maven, gestión de dependencias). \\
	\textbf{Responsable:} Guillermo Galavís \\
	\textbf{Estimación:} 4 horas
	\item Diseño y creación del esquema inicial de la base de datos (tablas de Usuarios, Roles, Solicitudes de Registro). \\
	\textbf{Responsable:} Guillermo Galavís \\
	\textbf{Estimación:} 6 horas
	\item Implementación base del Servicio de Reconocimiento Facial. \\
	\textbf{Responsable:} Renzo Morales \\
	\textbf{Estimación:} 3 horas
\end{itemize}

\textbf{Historia de Usuario: HU001 - Inicio de Sesión}
\begin{itemize}
	\item Diseñar y prototipar la interfaz de usuario para el inicio de sesión. \\
	\textbf{Responsable:} Guillermo Galavís \\
	\textbf{Estimación:} 3 horas
	\item Implementar la lógica de autenticación y diferenciación de roles de usuario/administrador. \\
	\textbf{Responsable:} Guillermo Galavís \\
	\textbf{Estimación:} 6 horas
	\item Escribir y ejecutar pruebas unitarias para la funcionalidad de inicio de sesión. \\
	\textbf{Responsable:} José Apure \\
	\textbf{Estimación:} 3 horas
\end{itemize}

\textbf{Historia de Usuario: HU002 - Solicitud de Cita para Registro}
\begin{itemize}
	\item Diseñar y prototipar el formulario de solicitud de cita. \\
	\textbf{Responsable:} José Apure \\
	\textbf{Estimación:} 3 horas
	\item Implementar la lógica de validación de campos del formulario (incluyendo email válido, cédula de 8 dígitos, campos condicionales). \\
	\textbf{Responsable:} José Apure \\
	\textbf{Estimación:} 6 horas
	\item Implementar el envío de datos de la solicitud al backend y manejo de respuestas. \\
	\textbf{Responsable:} José Apure \\
	\textbf{Estimación:} 4 horas
	\item Escribir y ejecutar pruebas unitarias para la funcionalidad de solicitud de cita. \\
	\textbf{Responsable:} Renzo Morales \\
	\textbf{Estimación:} 3 horas
\end{itemize}

\pagebreak

\subsection{Planificación del Sprint 2 (07/07/2025 - 13/07/2025)}

\subsubsection{Objetivos del Sprint 2}
\begin{itemize}
	\item Completar el flujo de registro de usuarios desde la perspectiva del administrador (HU003).
	\item Habilitar la gestión básica de horarios y menús (HU004 - Fase 1).
	\item Sentar las bases para la consulta de saldo del monedero virtual.
\end{itemize}

\subsubsection{Tareas del Sprint 2}
\textbf{Historia de Usuario: HU003 - Registro Físico de Usuario por Administrador}
\begin{itemize}
	\item Diseñar y prototipar la interfaz de administración para el procesamiento de solicitudes de registro. \\
	\textbf{Responsable:} Renzo Morales \\
	\textbf{Estimación:} 5 horas
	\item Implementar la lógica de visualización, validación y modificación de datos de solicitud por parte del administrador. \\
	\textbf{Responsable:} Renzo Morales \\
	\textbf{Estimación:} 8 horas
	\item Integración con la simulación de API de escaneo facial para el proceso de registro biométrico. \\
	\textbf{Responsable:} Renzo Morales \\
	\textbf{Estimación:} 8 horas
	\item Implementar la lógica de persistencia de usuarios registrados en la base de datos (actualización de solicitud a usuario activo). \\
	\textbf{Responsable:} Renzo Morales \\
	\textbf{Estimación:} 6 horas
	\item Escribir y ejecutar pruebas unitarias para la funcionalidad de registro por administrador. \\
	\textbf{Responsable:} Luis David Lima \\
	\textbf{Estimación:} 6 horas
\end{itemize}

\pagebreak

\textbf{Historia de Usuario: HU004 - Gestión de Horarios y Asignación de Menús (Fase 1: Horarios y Platos base)}
\begin{itemize}
	\item Diseño y extensión de la base de datos para Horarios, Platos, Menús (estructura básica). \\
	\textbf{Responsable:} José Apure \\
	\textbf{Estimación:} 5 horas
	\item Diseñar y prototipar la interfaz de gestión de horarios y platos. \\
	\textbf{Responsable:} José Apure \\
	\textbf{Estimación:} 6 horas
	\item Implementar la lógica para agregar, modificar y eliminar turnos de comedor y platos básicos. \\
	\textbf{Responsable:} José Apure \\
	\textbf{Estimación:} 10 horas
	\item Implementar la lógica de validación de superposición de turnos. \\
	\textbf{Responsable:} Renzo Morales \\
	\textbf{Estimación:} 3 horas
	\item Escribir y ejecutar pruebas unitarias para HU004 (Fase 1). \\
	\textbf{Responsable:} Guillermo Galavís \\
	\textbf{Estimación:} 5 horas
\end{itemize}

\textbf{Tareas Adicionales del Sprint 2}

\begin{itemize}
	\item Diseño y creación del esquema de base de datos para Monederos y Transacciones. \\
	\textbf{Responsable:} Guillermo Galavís \\
	\textbf{Estimación:} 4 horas
	\item Implementación de la lógica inicial para cargar y mostrar el saldo actual del monedero virtual. \\
	\textbf{Responsable:} Guillermo Galavís \\
	\textbf{Estimación:} 4 horas
\end{itemize}

\pagebreak

\subsection{Planificación del Sprint 3 (14/07/2025 - 20/07/2025)}

\subsubsection{Objetivos del Sprint 3}
\begin{itemize}
	\item Permitir a los usuarios visualizar los menús disponibles (HU006).
	\item Habilitar la consulta completa de saldo y movimientos del monedero virtual (HU012) y la visualización de datos para recarga (HU013).
	\item Completar la gestión de menús con componentes (HU004 - Fase 2).
	\item Implementar la configuración de tarifas diferenciadas (HU015).
\end{itemize}

\subsubsection{Tareas del Sprint 3}
\textbf{Historia de Usuario: HU006 - Visualización del Menú Diario}
\begin{itemize}
	\item Diseñar y prototipar la interfaz de usuario para el menú diario. \\
	\textbf{Responsable:} Guillermo Galavís \\
	\textbf{Estimación:} 3 horas
	\item Implementar la lógica para cargar y mostrar los menús disponibles del día desde el backend. \\
	\textbf{Responsable:} Guillermo Galavís \\
	\textbf{Estimación:} 6 horas
	\item Escribir y ejecutar pruebas unitarias para la funcionalidad de visualización del menú. \\
	\textbf{Responsable:} Renzo Morales \\
	\textbf{Estimación:} 4 horas
\end{itemize}

\textbf{Historia de Usuario: HU012 - Consulta de Saldo y Movimientos del Monedero Virtual}
\begin{itemize}
	\item Diseñar y prototipar la interfaz de usuario para la consulta de saldo y el historial de movimientos. \\
	\textbf{Responsable:} Guillermo Galavís \\
	\textbf{Estimación:} 4 horas
	\item Implementar la lógica para cargar y mostrar el saldo actual y los movimientos del monedero desde la base de datos. \\
	\textbf{Responsable:} Luis David Lima \\
	\textbf{Estimación:} 10 horas
	\item Desarrollo de API REST para la consulta de monedero (saldo y movimientos). \\
	\textbf{Responsable:} José Apure \\
	\textbf{Estimación:} 6 horas
	\item Escribir y ejecutar pruebas unitarias para la funcionalidad de consulta de monedero. \\
	\textbf{Responsable:} Luis David Lima \\
	\textbf{Estimación:} 4 horas
\end{itemize}

\textbf{Historia de Usuario: HU013 - Visualización de Datos para Recarga por Pago Móvil}
\begin{itemize}
	\item Diseñar y prototipar la interfaz de usuario para la sección de recarga por pago móvil. \\
	\textbf{Responsable:} José Apure \\
	\textbf{Estimación:} 3 horas
	\item Implementar la lógica para mostrar los datos de recarga (número, RIF, banco, etc.). \\
	\textbf{Responsable:} Guillermo Galavís \\
	\textbf{Estimación:} 4 horas
	\item Implementar NFR: Lógica de validación del límite de saldo de 5000 BS y simulación del «rebote» del dinero excedente. \\
	\textbf{Responsable:} Guillermo Galavís \\
	\textbf{Estimación:} 6 horas
	\item Escribir y ejecutar pruebas unitarias para la funcionalidad de recarga (visualización y validación de límite). \\
	\textbf{Responsable:} Luis David Lima \\
	\textbf{Estimación:} 4 horas
\end{itemize}

\textbf{Historia de Usuario: HU004 - Gestión de Horarios y Asignación de Menús (Fase 2: Gestión de Componentes de Platos)}
\begin{itemize}
	\item Diseñar y prototipar la interfaz para la gestión de componentes de platos. \\
	\textbf{Responsable:} Renzo Morales \\
	\textbf{Estimación:} 4 horas
	\item Implementar la lógica para asociar componentes a platos y gestión de insumos. \\
	\textbf{Responsable:} José Apure \\
	\textbf{Estimación:} 8 horas
	\item Escribir y ejecutar pruebas unitarias para HU004 (Fase 2). \\
	\textbf{Responsable:} José Apure \\
	\textbf{Estimación:} 4 horas
\end{itemize}

\pagebreak

\textbf{Historia de Usuario: HU015 - Configuración de Tarifas Diferenciadas}
\begin{itemize}
	\item Diseñar y prototipar la interfaz de configuración de tarifas. \\
	\textbf{Responsable:} Renzo Morales \\
	\textbf{Estimación:} 4 horas
	\item Implementar lógica para guardar y recuperar porcentajes de tarifas (estudiante, profesor, empleado). \\
	\textbf{Responsable:} José Apure \\
	\textbf{Estimación:} 4 horas
	\item Implementar lógica de validación de rangos para porcentajes de tarifas. \\
	\textbf{Responsable:} Luis David Lima \\
	\textbf{Estimación:} 3 horas
	\item Escribir y ejecutar pruebas unitarias para HU015. \\
	\textbf{Responsable:} Guillermo Galavís \\
	\textbf{Estimación:} 3 horas
\end{itemize}

\pagebreak

\subsection{Planificación del Sprint 4 (21/07/2025 - 27/07/2025)}

\subsubsection{Objetivos del Sprint 4}
\begin{itemize}
	\item Implementar el proceso completo de pago de comida y la generación/gestión/canje de códigos QR (HU009, HU010).
	\item Desarrollar la funcionalidad de reembolso automático (HU011).
	\item Gestionar los costos operacionales y cálculo del CCB (HU014).
	\item Desarrollar la generación y exportación de reportes operacionales para administradores (HU016).
\end{itemize}

\subsubsection{Tareas del Sprint 4}
\textbf{Historia de Usuario: HU009 - Pago Seguro de Comida y Generación de Factura QR}
\begin{itemize}
	\item Diseñar y prototipar la interfaz de usuario para el proceso de pago (reconocimiento facial, selección de menú, confirmaciones, indicador, QR). \\
	\textbf{Responsable:} Guillermo Galavís \\
	\textbf{Estimación:} 8 horas
	\item Diseño y extensión de la base de datos para registrar QRs generados, su estado y pagos asociados. \\
	\textbf{Responsable:} Guillermo Galavís \\
	\textbf{Estimación:} 3 horas
	\item Implementar la lógica de procesamiento de pago (descuento de saldo, múltiples confirmaciones, indicador de procesamiento con cancelación). \\
	\textbf{Responsable:} José Apure \\
	\textbf{Estimación:} 12 horas
	\item Implementar la generación del código QR único e irrepetible y su persistencia hasta el uso o fin del turno. \\
	\textbf{Responsable:} José Apure \\
	\textbf{Estimación:} 10 horas
	\item Escribir y ejecutar pruebas unitarias exhaustivas para la funcionalidad de pago y QR. \\
	\textbf{Responsable:} Renzo Morales \\
	\textbf{Estimación:} 10 horas
\end{itemize}

\pagebreak

\textbf{Historia de Usuario: HU010 - Visualización y Canje de Facturas QR}
\begin{itemize}
	\item Diseñar y prototipar la interfaz de usuario para visualizar lista de facturas con QR de pagos realizados. \\
	\textbf{Responsable:} Guillermo Galavís \\
	\textbf{Estimación:} 5 horas
	\item Diseñar y prototipar la interfaz para el personal del comedor para escanear y validar/canjear QR. \\
	\textbf{Responsable:} Guillermo Galavís \\
	\textbf{Estimación:} 8 horas
	\item Implementar la lógica de validación, canje y actualización de estado de factura por QR. \\
	\textbf{Responsable:} Guillermo Galavís \\
	\textbf{Estimación:} 10 horas
	\item Escribir y ejecutar pruebas unitarias para la funcionalidad de visualización y canje de facturas. \\
	\textbf{Responsable:} José Apure \\
	\textbf{Estimación:} 6 horas
\end{itemize}

\textbf{Historia de Usuario: HU011 - Reembolso Automático de Facturas No Canjeadas}
\begin{itemize}
	\item Diseño e implementación del job de reembolso automático para facturas no canjeadas en una semana. \\
	\textbf{Responsable:} Luis David Lima \\
	\textbf{Estimación:} 8 horas
	\item Implementar lógica para identificar facturas no canjeadas y vencidas (más de una semana). \\
	\textbf{Responsable:} Renzo Morales \\
	\textbf{Estimación:} 6 horas
	\item Implementar lógica de procesamiento de reembolso (actualizar monedero virtual del usuario). \\
	\textbf{Responsable:} José Apure \\
	\textbf{Estimación:} 8 horas
	\item Implementar notificación al usuario sobre el reembolso realizado. \\
	\textbf{Responsable:} José Apure \\
	\textbf{Estimación:} 4 horas
	\item Escribir y ejecutar pruebas unitarias para la funcionalidad de reembolso automático. \\
	\textbf{Responsable:} Guillermo Galavís \\
	\textbf{Estimación:} 6 horas
\end{itemize}

\pagebreak

\textbf{Historia de Usuario: HU014 - Gestión de Costos Operacionales y Cálculo del CCB}
\begin{itemize}
	\item Diseñar y prototipar la interfaz para el registro y consulta de costos operacionales (fijos, variables, número de bandejas, porcentaje de merma). \\
	\textbf{Responsable:} Luis David Lima \\
	\textbf{Estimación:} 5 horas
	\item Diseño y extensión de la base de datos para datos de costos operacionales. \\
	\textbf{Responsable:} Luis David Lima \\
	\textbf{Estimación:} 3 horas
	\item Implementar la lógica de cálculo del Costo Cubierto de la Bandeja (CCB). \\
	\textbf{Responsable:} José Apure \\
	\textbf{Estimación:} 6 horas
	\item Implementar la lógica para consulta y visualización del historial del CCB. \\
	\textbf{Responsable:} Renzo Morales \\
	\textbf{Estimación:} 5 horas
	\item Escribir y ejecutar pruebas unitarias para la funcionalidad de gestión de costos y CCB. \\
	\textbf{Responsable:} José Apure \\
	\textbf{Estimación:} 4 horas
\end{itemize}

\textbf{Historia de Usuario: HU016 - Generación y Exportación de Reportes Operacionales}
\begin{itemize}
	\item Diseñar y prototipar la interfaz de administración para la generación de reportes (selección de período, opciones de exportación). \\
	\textbf{Responsable:} José Apure \\
	\textbf{Estimación:} 4 horas
	\item Implementar la lógica para la agregación y procesamiento de datos para los diferentes tipos de reportes (diario, semanal, mensual). \\
	\textbf{Responsable:} Renzo Morales \\
	\textbf{Estimación:} 8 horas
	\item Implementar la funcionalidad de exportación de reportes a formato de hoja de cálculo (ej. CSV o XLSX). \\
	\textbf{Responsable:} José Apure \\
	\textbf{Estimación:} 6 horas
	\item Escribir y ejecutar pruebas unitarias para la funcionalidad de generación de reportes. \\
	\textbf{Responsable:} Luis David Lima \\
	\textbf{Estimación:} 5 horas
\end{itemize}

\pagebreak

\subsection{Definición de Hecho (Definition of Done)}

Para este Sprint, una historia de usuario se considerará «hecha» cuando se cumplan todos los siguientes criterios:

\begin{itemize}
	\item El código de la funcionalidad ha sido implementado y cumple con todos los criterios de aceptación de la historia de usuario.
	\item Todas las pruebas unitarias asociadas a la funcionalidad han sido escritas y pasan el 100\%.
	\item El código ha sido revisado por al menos otro miembro del equipo (Pair Programming o Code Review).
	\item El código ha sido integrado exitosamente en la rama principal del repositorio.
	\item La funcionalidad es demostrable al Grupo Docente.
\end{itemize}

\end{document}
